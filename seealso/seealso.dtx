% \iffalse meta-comment
%
% Copyright (C) 2014 by Leo Liu <leoliu.pku@gmail.com>
% ---------------------------------------------------------------------------
% This work may be distributed and/or modified under the
% conditions of the LaTeX Project Public License, either version 1.3
% of this license or (at your option) any later version.
% The latest version of this license is in
%   http://www.latex-project.org/lppl.txt
% and version 1.3 or later is part of all distributions of LaTeX
% version 2005/12/01 or later.
%
% This work has the LPPL maintenance status `maintained'.
%
% The Current Maintainer of this work is Leo Liu.
%
% This work consists of the files seealso.dtx and seealso.ins
% and the derived filebase seealso.sty.
%
% \fi
%
% \iffalse
%<*driver>
\ProvidesFile{seealso.dtx}
%</driver>
%<package>\NeedsTeXFormat{LaTeX2e}[1999/12/01]
%<package>\ProvidesPackage{seealso}
%<*package>
    [2014/04/04 v0.1 makeidx's see and seealso with page number support.]
%</package>
%
%<*driver>
\documentclass{ltxdoc}
\newcommand\pkg[1]{\textsf{#1}}
\usepackage[UTF8,hyperref]{ctex}
\makeatletter
\renewcommand\glossary@prologue{%
  \section*{版本历史}
  \markboth{版本历史}{版本历史}}
\renewcommand\index@prologue{%
  \section*{Index / 代码索引}
  \markboth{Index / 代码索引}{Index / 代码索引}
  斜体的数字表示对应项说明所在的页码。下划线的数字表示定义所在的代码行号;而直
  立体的数字表示对应项使用时所在的行号。}
\def\option{\begingroup
   \catcode`\\12
   \MakePrivateLetters \option@}
\let\endoption\endtrivlist
\long\def\option@#1{\endgroup \topsep\MacroTopsep \trivlist
  \edef\saved@macroname{\string#1}%
  \def\makelabel##1{\llap{##1}}%
  \if@inlabel
    \let\@tempa\@empty \count@\macro@cnt
    \loop \ifnum\count@>\z@
      \edef\@tempa{\@tempa\hbox{\strut}}\advance\count@\m@ne \repeat
    \edef\makelabel##1{\llap{\vtop to\baselineskip
                               {\@tempa\hbox{##1}\vss}}}%
    \advance \macro@cnt \@ne
  \else  \macro@cnt\@ne  \fi
  \edef\@tempa{\noexpand\item[%
     \noexpand\PrintEnvName
     {\string#1}]}%
  \@tempa
  \global\advance\c@CodelineNo\@ne
      \SpecialMainOptionIndex{#1}\nobreak
  \global\advance\c@CodelineNo\m@ne
  \ignorespaces}
\def\SpecialMainOptionIndex#1{\@bsphack
  \special@index{%
    #1\actualchar
    {\string\ttfamily\space#1}
    (option)%
    \encapchar main}%
  \@esphack}
\makeatother
\usepackage[override,activecr]{seealso}
\usepackage{hypdoc}
\AtBeginDocument{\MakeShortVerb\|}
\EnableCrossrefs
\CodelineIndex
\RecordChanges
\begin{document}
  \DocInput{seealso.dtx}
  \PrintChanges
  \PrintIndex
\end{document}
%</driver>
% \fi
%
% \CheckSum{125}
%
% \CharacterTable
%  {Upper-case    \A\B\C\D\E\F\G\H\I\J\K\L\M\N\O\P\Q\R\S\T\U\V\W\X\Y\Z
%   Lower-case    \a\b\c\d\e\f\g\h\i\j\k\l\m\n\o\p\q\r\s\t\u\v\w\x\y\z
%   Digits        \0\1\2\3\4\5\6\7\8\9
%   Exclamation   \!     Double quote  \"     Hash (number) \#
%   Dollar        \$     Percent       \%     Ampersand     \&
%   Acute accent  \'     Left paren    \(     Right paren   \)
%   Asterisk      \*     Plus          \+     Comma         \,
%   Minus         \-     Point         \.     Solidus       \/
%   Colon         \:     Semicolon     \;     Less than     \<
%   Equals        \=     Greater than  \>     Question mark \?
%   Commercial at \@     Left bracket  \[     Backslash     \\
%   Right bracket \]     Circumflex    \^     Underscore    \_
%   Grave accent  \`     Left brace    \{     Vertical bar  \|
%   Right brace   \}     Tilde         \~}
%
%
% \changes{v0.1}{2014/04/04}{初始版本。}
%
% \DoNotIndex{\newcommand,\newenvironment,\^,\~,\active,\begingroup,\catcode,
% \csdef,\cslet,\cslet,\csuse,\DeclareBoolOption,\def,\do,\dolistloop,\else,
% \emph,\empty,\endgroup,\fi,\forlistcsloop,\forlistloop,\ifcsdef,\ifcsempty,
% \ifinlistcs,\lccode,\let,\listadd,\listcsgadd,\lowercase,\newif,
% \ProcessKeyvalOptions,\RequirePackage,\SetupKeyvalOptions,\space,\unless}
%
% \GetFileInfo{seealso.dtx}
% \title{The \pkg{seealso} package}
% \author{Leo Liu \\ \nolinkurl{leoliu.pku@gmail.com}}
% \date{\fileversion~from \filedate}
%
% \maketitle
%
% \section{Introduction}
%
% Put text here.
%
% \seepage{AAA}{5}, \seepage{BBB}{6}, \seepage{AAA}{6}, \seepage{CCC}{7}
%
% \seealso{AAA}{5}, \seealso{BBB}{6}, \seealso{CCC}{7}
%
% \section{Usage}
%
% Put text here.
%
%
% \StopEventually{}
%
% \section{Implementation / 代码实现}
%
% \iffalse
%<*package>
% \fi
%
% 引入相关编程工具。
%    \begin{macrocode}
\RequirePackage{etoolbox}
\RequirePackage{kvoptions}
\SetupKeyvalOptions{
  family=seealso@opt,
  prefix=seealso@}
%    \end{macrocode}
%
% 声明宏包选项。
%
% \begin{option}{override}
% 重定义 |\see| 与 |\seealso| 为有页码的形式。
%    \begin{macrocode}
\DeclareBoolOption{override}
%    \end{macrocode}
% \end{option}
%
% \begin{option}{activecr}
% 使用换行符作为输出 |\see| 等命令的指令。
%    \begin{macrocode}
\DeclareBoolOption{activecr}
%    \end{macrocode}
% \end{option}
%
% 执行选项。
%    \begin{macrocode}
\ProcessKeyvalOptions*
%    \end{macrocode}
%
% \begin{macro}{\seealso@charlet}
% 参数 |#1| 是一个字符或 |\| 加字符的形式,|\seealso@charlet| 将此字符看做活动
% 字符的宏,使用 |\let| 与后面的内容赋值,但本身不改变字符的 catcode。
%    \begin{macrocode}
\def\seealso@charlet#1{%
  \begingroup\lccode`\~=`#1\lowercase{\endgroup\let~}}
%    \end{macrocode}
% \end{macro}
%
% \begin{macro}{\seealso@macrolist}
% 列表记录所有独立的类似 |\seepage| 的宏。默认只有 |see| 和 |also| 两组,对应
% 命令 |\seepage| 和 |\seealsopage|。
%    \begin{macrocode}
\let\seealso@macrolist\empty
%    \end{macrocode}
% \end{macro}
%
% \begin{macro}{\seealso@clearlists}
% 清空列表 |\seealso@see@list| 与 |\seealso@also@list|。
%    \begin{macrocode}
\def\seealso@clearlists{%
  \def\do##1{\cslet{seealso@##1@list}\empty}%
  \dolistloop\seealso@macrolist}
%    \end{macrocode}
% \end{macro}
% 
% \begin{macro}{\SeealsoPrintList}
% 输出列表 |\seealso@see@list| 与 |\seealso@also@list|。
%    \begin{macrocode}
\newcommand\SeealsoPrintList{%
  \forlistloop\seealso@printlist\seealso@macrolist}
%    \end{macrocode}
% \end{macro}
% 
% \begin{macro}{\ifseealso@firstitem}
% 测试是否是在输出参见列表的第一项。
%    \begin{macrocode}
\newif\ifseealso@firstitem
%    \end{macrocode}
% \end{macro}
%
% \begin{macro}{\seealso@printlist}
% 输出参见列表 |#1|。如果列表为空则无操作。
%    \begin{macrocode}
\def\seealso@printlist#1{%
  \ifcsempty{seealso@#1@list}
    {}
    {\csuse{seealso@#1@listsep}%
     \seealso@nameformat{\csuse{#1name}}%
     \seealso@firstitemtrue
     \forlistcsloop{\seealso@listitem}{seealso@#1@list}}}
%    \end{macrocode}
% \end{macro}
%
% \begin{macro}{\seealso@listitem}
% 输出参见列表的一项。如果不是第一项,同时输出分隔符。
%    \begin{macrocode}
\def\seealso@listitem#1{%
  \ifseealso@firstitem
    \seealso@firstitemfalse
  \else
    \seealso@itemsep
  \fi
  \seealso@itemformat{#1}}
%    \end{macrocode}
% \end{macro}
% 
% \begin{macro}{\DeclareSeealsoMacro}
% 定义一个新的带页码的参见命令。|#1| 是命令名,|#2| 是该命令使用的参见列
% 表,|#3| 是列表输出时使用的名字。
%    \begin{macrocode}
\newcommand\DeclareSeealsoMacro[3]{%
  \newcommand#1[2]{%
    \seealso@setactivecr
    \ifinlistcs{##1}{seealso@#2@list}
      {}
      {\listcsgadd{seealso@#2@list}{##1}}%
    \seealso@pageformat{##2}}
  \listadd\seealso@macrolist{#2}%
  \ifcsdef{#2name}
    {}
    {\csdef{#2name}{#3}}%
  \csdef{seealso@#2@listsep}{\seealso@listsep}%
  \csdef{seealso@#2@itemsep}{\seealso@itemsep}%
  \csdef{seealso@#2@nameformat}{\seealso@nameformat}%
  \csdef{seealso@#2@itemformat}{\seealso@itemformat}%
  \csdef{seealso@#2@pageformat}{\seealso@pageformat}%
}
%    \end{macrocode}
% \end{macro}
%
% 设置参见列表的输出格式。以下宏给出的都是全局的默认值,可以单独修改每一项的输
% 出格式。
% \begin{macro}{\seealso@listsep}
% 参见列表之前的分隔符。
%    \begin{macrocode}
\def\seealso@listsep{;\space}
%    \end{macrocode}
% \end{macro}
% \begin{macro}{\seealso@itemsep}
% 参见列表项之间的分隔符。
%    \begin{macrocode}
\def\seealso@itemsep{,\space}
%    \end{macrocode}
% \end{macro}
% \begin{macro}{\seealso@nameformat}
% 参见名“see also”的输出格式。
%    \begin{macrocode}
\def\seealso@nameformat#1{\emph{#1}\space}
%    \end{macrocode}
% \end{macro}
% \begin{macro}{\seealso@itemformat}
% 参见列表项的输出格式。
%    \begin{macrocode}
\def\seealso@itemformat#1{#1}
%    \end{macrocode}
% \end{macro}
% \begin{macro}{\seealso@pageformat}
% 参见页码的输出格式。
%    \begin{macrocode}
\def\seealso@pageformat#1{#1}
%    \end{macrocode}
% \end{macro}
%
% \begin{macro}{\ifseealso@iscractive}
% 判断当前换行符是否已经被激活。
%    \begin{macrocode}
\newif\ifseealso@iscractive
%    \end{macrocode}
% \end{macro}
% 
% \begin{macro}{\seealso@setactivecr}
% |activecr| 选项为真时,设置换行符为 |\seealso@cr| 并激活。
%    \begin{macrocode}
\def\seealso@setactivecr{%
  \ifseealso@activecr
    \unless\ifseealso@iscractive
      \seealso@clearlists
      \catcode`\^^M=\active
      \seealso@charlet\^^M\seealso@cr
      \seealso@iscractivetrue
    \fi
  \fi}
%    \end{macrocode}
% \end{macro}
% 
% \begin{macro}{\seealso@cr}
% 是在 |activecr| 选项下,换行符的定义。它取消激活换行符为宏,并输出参见列表。
%    \begin{macrocode}
\def\seealso@cr{%
  \catcode`\^^M=5
  \seealso@iscractivefalse
  \SeealsoPrintList}
%    \end{macrocode}
% \end{macro}
%
% \begin{macro}{\seenopage}
% \begin{macro}{\seealsonopage}
% 保存旧的 |\see| 与 |\seealso| 命令定义,使用 |override| 选项时可临时使用旧的
% 定义。
%    \begin{macrocode}
\let\seenopage\see
\let\seealsonopage\seealso
%    \end{macrocode}
% \end{macro}
% \end{macro}
%
% \begin{macro}{\seepage}
% \begin{macro}{\seealsopage}
% 带页码输出的 |\seepage| 与 |\seealsopage|。使用 |override| 选项时可直接使用
% |\see| 与 |\seealso| 代替 |\seepage| 与 |\seealsopage|。
%
% 如果之前没有定义,这里会同时定义 |\seename| 和 |\alsoname|,与 \pkg{makeidx}
% 初始值一致。
%    \begin{macrocode}
\DeclareSeealsoMacro\seepage{see}{see}
\DeclareSeealsoMacro\seealsopage{also}{see also}
%    \end{macrocode}
% \end{macro}
% \end{macro}
%
% \begin{macro}{\see}
% \begin{macro}{\seealso}
% 使用 |override| 时,重定义 |\see| 与 |\seealso|。
%    \begin{macrocode}
\ifseealso@override
  \def\see{\seepage}
  \def\seealso{\seealsopage}
\fi
%    \end{macrocode}
% \end{macro}
% \end{macro}
% 
% \iffalse
%</package>
% \fi
%
% \Finale
\endinput
