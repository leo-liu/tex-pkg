% \iffalse meta-comment
%
% Copyright (C) 2012 by Leo Liu <leoliu.pku@gmail.com>
% ---------------------------------------------------------------------------
% This work may be distributed and/or modified under the
% conditions of the LaTeX Project Public License, either version 1.3
% of this license or (at your option) any later version.
% The latest version of this license is in
%   http://www.latex-project.org/lppl.txt
% and version 1.3 or later is part of all distributions of LaTeX
% version 2005/12/01 or later.
%
% This work has the LPPL maintenance status `maintained'.
%
% The Current Maintainer of this work is Leo Liu.
%
% This work consists of the files zhmCJK.dtx and zhmCJK.ins
% with the derived filebase zhmCJK.sty, the font mapping files
% zhmCJK.map and texfonts.map, and the test code test-zhmCJK.tex.
%
% \fi
%
% \iffalse
%<*driver>
\ProvidesFile{zhmCJK.dtx}
%</driver>
%<package>\NeedsTeXFormat{LaTeX2e}[1999/12/01]
%<package>\ProvidesPackage{zhmCJK}
%<*package>
    [2012/02/02 v0.3 setup CJK fonts with zhmetrics]
%</package>
%
%<*driver>
\documentclass[11pt,a4paper,
  driverfallback=dvipdfmx,unicode]{ltxdoc}
\usepackage[numbered]{hypdoc}
\hypersetup{pdfstartview=FitH}
\usepackage{zhmCJK}
\setCJKmainfont{simsun.ttc}
\setCJKsansfont{simhei.ttf}
\setCJKmonofont{simfang.ttf}
\usepackage{longtable}
\setlength\parindent{2em}
\linespread{1.2}
\makeatletter
\renewcommand\glossary@prologue{%
  \section*{版本历史}
  \markboth{版本历史}{版本历史}}
\renewcommand\index@prologue{%
  \section*{代码索引}
  \markboth{代码索引}{代码索引}
  斜体的数字表示对应项说明所在的页码。下划线的数字表示定义所在的代码行号;而直
  立体的数字表示对应项使用时所在的行号。}
\makeatother
\newcommand\eTeX{$\varepsilon$-\TeX}
\newcommand\pkg[1]{\textsf{#1}}
\AtBeginDocument{
  \DeleteShortVerb{\|}
  \MakeShortVerb{\"}}
\EnableCrossrefs
\CodelineIndex
\RecordChanges
\begin{document}
  \DocInput{zhmCJK.dtx}
  \PrintChanges
  \PrintIndex
\end{document}
%</driver>
% \fi
%
% \CheckSum{161}
%
% \CharacterTable
%  {Upper-case    \A\B\C\D\E\F\G\H\I\J\K\L\M\N\O\P\Q\R\S\T\U\V\W\X\Y\Z
%   Lower-case    \a\b\c\d\e\f\g\h\i\j\k\l\m\n\o\p\q\r\s\t\u\v\w\x\y\z
%   Digits        \0\1\2\3\4\5\6\7\8\9
%   Exclamation   \!     Double quote  \"     Hash (number) \#
%   Dollar        \$     Percent       \%     Ampersand     \&
%   Acute accent  \'     Left paren    \(     Right paren   \)
%   Asterisk      \*     Plus          \+     Comma         \,
%   Minus         \-     Point         \.     Solidus       \/
%   Colon         \:     Semicolon     \;     Less than     \<
%   Equals        \=     Greater than  \>     Question mark \?
%   Commercial at \@     Left bracket  \[     Backslash     \\
%   Right bracket \]     Circumflex    \^     Underscore    \_
%   Grave accent  \`     Left brace    \{     Vertical bar  \|
%   Right brace   \}     Tilde         \~}
%
%
% \changes{v0.1}{2012/02/02}{初始版本}
% \changes{v0.2}{2012/02/02}{编写宏包文档。增加 \pkg{CJKpunct}。做一些小的代码
% 调整。}
%
% \DoNotIndex{\newcommand,\newenvironment,
%   \@ifpackageloaded, \@ne, \@onlypreamble, \@xxxii, \advance, \aftergroup,
%   \AtBeginDocument, \AtBeginDvi, \AtBeginShipoutFirst, \AtEndDocument,
%   \AtEndOfPackage, \begin, \clearpage, \DeclareBoolOption,
%   \DeclareRobustCommand, \def, \else, \end, \endinput, \fi, \font, \ifnum,
%   \ifpdf, \let, \m@ne, \mathrm, \mathsf, \mathtt, \newcount,
%   \not@math@alphabet, \ProcessKeyvalOptions, \providecommand, \relax,
%   \RequirePackage, \scantokens, \SetupKeyvalOptions, \the}
%
% \GetFileInfo{zhmCJK.dtx}
% \title{\pkg{zhmCJK} 宏包}
% \author{刘海洋 \\ \nolinkurl{leoliu.pku@gmail.com}}
% \date{\filedate\ \fileversion}
%
% \maketitle
%
% \section{简介}
%
% \pkg{zhmCJK} 宏包是一个基于 \pkg{zhmetrics} 和 \pkg{CJK} 宏包的 CJK 文字配置
% 宏包,它为 \pkg{zhmetrics} 提供了一个方便的 \LaTeX{} 用户界面,可以仅指定字
% 体文件名,就完成原来十分复杂的 CJK 字体安装设置工作。
%
% \pkg{zhmCJK} 提供了尽可能简单的用户界面。除了提供 UTF-8 编码下对 \pkg{CJK}
% 宏包所用字体的实时安装设置功能,\pkg{zhmCJK} 还同时加载了 \pkg{CJKpunct} 和
% \pkg{CJKspace} 宏包处理标点压缩和字符间距。
%
% \pkg{zhmCJK} 支持 pdf\TeX{} 和 DVIPDFMx 两种输出驱动,可以使用 "pdflatex" 或
% $"latex"+"dvipdfmx"$ 的方式编译。使用 \pkg{zhmCJK} 需要 \eTeX{} 支持,并依赖
% \pkg{CJK}, \pkg{CJKpunct}, \pkg{CJKspace} 和 \pkg{ifpdf} 宏包。
%
% \section{用法}
%
% \subsection{宏包载入与选项}
%
% 宏包提供的选项如下:
% \begin{longtable}{l|l|l|l}
% \hline
% 选项 & 取值 & 默认值 & 功能 \\
% \hline
% "pdffakebold" & "true|false" & "true" & 使用 PDF 原语生成伪粗体 \\
% \hline
% \end{longtable}
%
% 只需要在导言区使用 "\usepackage{zhmCJK}" 载入宏包即可载入宏包。宏包可以带有
% 一些选项,例如如果要使用 \pkg{CJK} 通过重复输出得到的伪粗体,就可以用
% \begin{verbatim}
% \usepackage[pdffakebold=false]{zhmCJK}
% \end{verbatim}
% 一般来说使用默认的选项即可。
%
% \subsection{命令}
%
% \pkg{zhmCJK} 的基本用户界面与 \pkg{xeCJK} 十分相似,定义字体的几个命令语法大
% 体相同。
%
% \DescribeMacro{\setCJKmainfont}
% 设置正文默认罗马族的 CJK 字体,字体用 TrueType 文件名表示。该命令影响
% "\rmfamily" 和 "\textfm" 的字体。
%
% \DescribeMacro{\setCJKromanfont}
% 是 "\setCJKmainfont" 的别名。
%
% 例如,使用
% \begin{verbatim}
% \setCJKmainfont{simsun.ttc}
% \end{verbatim}
% 将使用文件名为 "simsun.ttc" 的字体(中易宋体)作为正文罗马族字体。
%
% \DescribeMacro{\setCJKsansfont}
% 设置正文无衬线族的 CJK 字体。影响 "\sffamily" 和 "\textsf" 的字体。
%
% \DescribeMacro{\setCJKmonofont}
% 设置正文等宽族的 CJK 字体。影响 "\ttfamily" 和 "\texttt" 的字体。
%
% \DescribeMacro{\setCJKfamilyfont}
% 定义新的 CJK 字体族并指定字体。例如用
% \begin{verbatim}
% \setCJKfamilyfont{yahei}{msyh.ttf}
% \end{verbatim}
% 可以定义名为 "yahei" 的 CJK 字体族,其对应的实际字体文件名为 "msyh.ttf"(微
% 软雅黑)。
%
% 以上字体设置命令只能在文档导言区使用。中文字体可以用 \pkg{CJK} 宏包的
% "\CJKfamily" 命令切换。例如用 "\CJKfamily{yahei}" 就可以选择前面定义的微软雅
% 黑字体。
%
%
% \DescribeMacro{\CJKrmdefault}
% 保存 "\rmfamily" 所使用的 CJK 字体族,默认值是 "rm"。类似西文字体的
% "\rmdefault"。
%
% \DescribeMacro{\CJKsfdefault} 保存 "\sffamily" 所使用的 CJK 字体族,默认值是
% "sf"。类似西文字体的 "\sfdefault"。
%
% \DescribeMacro{\CJKttdefault}
% 保存 "\ttfamily" 所使用的 CJK 字体族,默认值是 "tt"。类似西文字体的
% "\ttdefault"。
%
% \DescribeMacro{\CJKfamilydefault}
% 保存 "\normalfont" 所使用的 CJK 字体族,默认值是 "\CJKrmdefault"。类似西文字
% 体的 "\familydefault"。例如,使用
% \begin{verbatim}
% \renewcommand\familydefault{\sfdefault}
% \renewcommand\CJKfamilydefault{\CJKsfdefault}
% \end{verbatim}
% 可以将全文的 CJK 和西文默认字体改为无衬线字体族。
%
%
% \section{已知问题}
%
% \begin{itemize}
%  \item 暂时不提供复合字体,即将几个不同的实际字体定义为一个 CJK 字体族不同形
%  状的功能。
%  \item 受制于预定义的映射文件 "texfonts.map",使用 \pkg{zhmCJK} 在同一文档中
%  能够使用的字体文件是有限的。目前只有 32 个。
% \end{itemize}
%
% \StopEventually{}
%
% \clearpage
% \section{代码实现}
%
% \iffalse
%<*package>
% \fi
%
% 首先载入基本的 CJK 相关支持包和工具宏包。
%
% 基本 CJK 支持,UTF-8 编码。
%    \begin{macrocode}
\RequirePackage{CJKutf8}
%    \end{macrocode}
% CJK 字符与西文字符之间的空格。
%    \begin{macrocode}
\RequirePackage{CJKspace}
%    \end{macrocode}
% CJK 标点禁则与压缩。
%    \begin{macrocode}
\RequirePackage{CJKpunct}
%    \end{macrocode}
%
% 工具宏包。
%    \begin{macrocode}
\RequirePackage{ifpdf}
\RequirePackage{kvoptions}
\SetupKeyvalOptions{
  family=zhm,
  prefix=zhm@
}
%    \end{macrocode}
%
% \changes{v0.3}{2012/02/02}{\texttt{pdffakebold} 选项}
% "pdffakebold" 选项选择是否使用 PDF 原语生成伪粗体。默认是 "true",如果选
% "false" 则改用原来 \pkg{CJK} 宏包平移输出的伪粗体机制。
%    \begin{macrocode}
\DeclareBoolOption[true]{pdffakebold}
%    \end{macrocode}
%
% 执行选项。
%    \begin{macrocode}
\ProcessKeyvalOptions*
%    \end{macrocode}
%
% \begin{macro}{\zhm@pdfliteral}
% 插入 PDF 原语。
%    \begin{macrocode}
\ifpdf
  \def\zhm@pdfliteral#1{\pdfliteral direct {#1}}
\else
  \def\zhm@pdfliteral#1{\special{pdf: literal direct #1}}
\fi
%    \end{macrocode}
% \end{macro}
%
% 使用 PDF 原语生成伪粗体。
% \changes{v0.3}{2012/02/02}{使用 PDF 原语生成伪粗体}
%    \begin{macrocode}
\ifzhm@pdffakebold
%    \end{macrocode}
% \begin{macro}{\CJKbold}
%    \begin{macrocode}
  \def\CJKbold{\zhm@pdfliteral{q 2 Tr 0.4 w}%
      \aftergroup\CJKnormal}
%    \end{macrocode}
% \end{macro}
% \begin{macro}{\CJKnormal}
%    \begin{macrocode}
  \def\CJKnormal{\zhm@pdfliteral{0 Tr}}
\fi
%    \end{macrocode}
% \end{macro}
%
% 在导言区和正文中分别开启 \pkg{CJK} 的功能。
%    \begin{macrocode}
\AtEndOfPackage{\CJK@makeActive}
\AtBeginDocument{\begin{CJK*}{UTF8}{\CJKfamilydefault}}
\AtEndDocument{\clearpage\end{CJK*}}
%    \end{macrocode}
%
% \begin{macro}{\zhmfamcnt}
% 已定义的 CJK 字体数。
%    \begin{macrocode}
\newcount\zhmfamcnt
%    \end{macrocode}
% \end{macro}
%
% \begin{macro}{\zhm@definefont}
% 定义 CJK 字体族。模拟 |.fd| 文件的定义。
%    \begin{macrocode}
\def\zhm@definefont#1{
  \DeclareFontFamily{C70}{#1}{\hyphenchar\font\m@ne}
%    \end{macrocode}
% "\DeclareFontShape" 内部需要修改 "\catcode",这里使用 "\scantokens" 处理。
%    \begin{macrocode}
  \scantokens{
    \DeclareFontShape{C70}{#1}{m}{n}{<-> CJK * zhm\the\zhmfamcnt}{\CJKnormal}
    \DeclareFontShape{C70}{#1}{bx}{n}{<-> CJKb * zhm\the\zhmfamcnt}{\CJKbold}
  }
}
%    \end{macrocode}
% \end{macro}
%
% \begin{macro}{\zhm@mapline}
% 添加实际字体映射。为 pdf\TeX{} 与 DVIPDFMx 引擎使用不同的命令完成。
%    \begin{macrocode}
\ifpdf
  \def\zhm@mapline#1{%
    \pdfmapline{=zhm\the\zhmfamcnt @Unicode@ <#1}}
\else
  \def\zhm@mapline#1{%
    \special{pdf:mapline + zhm\the\zhmfamcnt @Unicode@ unicode #1}}
\fi
%    \end{macrocode}
% \end{macro}
%
% \begin{macro}{\setCJKfamilyfont}
% 设置一个 CJK 字体族。
%    \begin{macrocode}
\def\setCJKfamilyfont#1#2{
  \advance\zhmfamcnt\@ne
  \ifnum\zhmfamcnt>\@xxxii
    \PackageError{zhmCJK}%
      {No more CJK font families can be setup.}%
      {There are at most 32 families setup by zhmCJK.}
  \else
%    \end{macrocode}
% 载入字体族及字形定义。
%    \begin{macrocode}
    \zhm@definefont{#1}
%    \end{macrocode}
% 字体映射需要在输出例程初始处设置。
%    \begin{macrocode}
    \AtBeginDvi{\zhm@mapline{#2}}
%    \end{macrocode}
% 如果载入了 \pkg{atbegshi} 宏包,则还要处理修改了的输出例程。这会影响
% \pkg{eso-pic} 等用户层宏包。
%    \begin{macrocode}
    \AtBeginDocument{%
      \@ifpackageloaded{atbegshi}{\AtBeginShipoutFirst{%
        \zhm@mapline{#2}}}{}}
  \fi
}
\@onlypreamble\setCJKfamilyfont
%    \end{macrocode}
% \end{macro}
%
%
% \begin{macro}{\setCJKmainfont}
% 设置 CJK 普通(罗马)字体。
%    \begin{macrocode}
\def\setCJKmainfont{%
  \setCJKfamilyfont{\CJKrmdefault}}
\@onlypreamble\setCJKmainfont
%    \end{macrocode}
% \end{macro}
% \begin{macro}{\setCJKromanfont}
% \changes{v0.3}{2012/02/02}{新增。}
% "\setCJKmainfont" 的别名。
%    \begin{macrocode}
\let\setCJKromanfont\setCJKmainfont
\@onlypreamble\setCJKromanfont
%    \end{macrocode}
% \end{macro}
%
%
% \begin{macro}{\setCJKsansfont}
% 设置 CJK 无衬线字体。
%    \begin{macrocode}
\def\setCJKsansfont{%
  \setCJKfamilyfont{\CJKsfdefault}}
\@onlypreamble\setCJKsansfont
%    \end{macrocode}
% \end{macro}
%
%
% \begin{macro}{\setCJKmonofont}
% 设置 CJK 等宽(打字机)字体。
%    \begin{macrocode}
\def\setCJKmonofont{%
  \setCJKfamilyfont{\CJKttdefault}}
\@onlypreamble\setCJKmonofont
%    \end{macrocode}
% \end{macro}
%
%
% \begin{macro}{\CJKrmdefault}
% CJK 罗马体默认字体族,作用于 "\rmfamily"。
%    \begin{macrocode}
\providecommand\CJKrmdefault{rm}
%    \end{macrocode}
% \end{macro}
%
%
% \begin{macro}{\CJKsfdefault}
% CJK 无衬线体默认字体族,作用于 "\sffamily"。
%    \begin{macrocode}
\providecommand\CJKsfdefault{sf}
%    \end{macrocode}
% \end{macro}
%
%
% \begin{macro}{\CJKttdefault}
% CJK 打字机体默认字体族,作用于 "\ttfamily"。
%    \begin{macrocode}
\providecommand\CJKttdefault{tt}
%    \end{macrocode}
% \end{macro}
%
%
% \begin{macro}{\CJKfamilydefault}
% CJK 默认字体族,作用于 "\normalfont"。
%    \begin{macrocode}
\providecommand\CJKfamilydefault{\CJKrmdefault}
%    \end{macrocode}
% \end{macro}
%
% 重定义 "\normalfont", "\rmfamily", "\sffamily" 和 "\ttfamily",使其同时设置
% CJK 字体。
%    \begin{macrocode}
\DeclareRobustCommand\normalfont
        {\CJKfamily{\CJKfamilydefault}%
         \usefont\encodingdefault
                 \familydefault
                 \seriesdefault
                 \shapedefault
         \relax}
\let\reset@font\normalfont
\DeclareRobustCommand\rmfamily
        {\not@math@alphabet\rmfamily\mathrm
         \fontfamily\rmdefault\CJKfamily{\CJKrmdefault}\selectfont}
\DeclareRobustCommand\sffamily
        {\not@math@alphabet\sffamily\mathsf
         \fontfamily\sfdefault\CJKfamily{\CJKsfdefault}\selectfont}
\DeclareRobustCommand\ttfamily
        {\not@math@alphabet\ttfamily\mathtt
         \fontfamily\ttdefault\CJKfamily{\CJKttdefault}\selectfont}
\endinput
%    \end{macrocode}
%
% \iffalse
%</package>
% \fi
%
% \Finale
\endinput
