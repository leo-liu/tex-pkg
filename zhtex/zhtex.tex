%% This is file `zhtex.tex', a minimal Chinese support for Plain TeX.
%%
%% This file is for personal use only.
%% Only UTF-8 encoding is (partially) supported.
%% The zhspacing package is loaded under XeTeX,
%% otherwise a minimal support for CJK fonts is setup.
%%
%% Copyright (C) 2013 by Leo Liu <leoliu.pku@gmail.com>
\catcode`@=11

\def\ifZH@primitive#1{\begingroup
  \edef\tempa{\meaning#1}%
  \edef\tempb{\string#1}%
  \expandafter\endgroup
    \ifx\tempa\tempb}

\ifZH@primitive\eTeXversion\else
  \immediate\write16{zhtex Error: We need eTeX support for Chinese.}
  \expandafter\endinput
\fi

\ifZH@primitive\XeTeXrevision
  \input zhspacing.sty
  \zhspacing
  \expandafter\endinput
\fi

\def\ZH@hexdigit#1{\ifcase\numexpr#1\relax
 0\or 1\or 2\or 3\or 4\or 5\or 6\or 7\or
 8\or 9\or a\or b\or c\or d\or e\or f\fi}
\def\ZH@hexbyte#1{%
    \ZH@hexdigit{\numexpr (#1 - 8) / 16\relax}%
    \ZH@hexdigit{\numexpr #1 - (#1 - 8) / 16 * 16\relax}}

\def\ZH@defutftriple#1{%
  \begingroup\uccode`~="E#1\uppercase{\endgroup
    \def~##1##2{%
      \begingroup
        \ifnum`##1="80
          \def\ZH@firstbyte{\numexpr "#1 * 16\relax}%
          \def\ZH@secondbyte{\numexpr `##2 - "80\relax}%
        \else
          \def\ZH@firstbyte{\numexpr "#1 * 16 + (`##1 - "80 - 2) / 4\relax}%
          \def\ZH@secondbyte{\numexpr (`##1 - "80 - (`##1 - "80 - 2) / 4 * 4) * 64 + `##2 - "80\relax}%
        \fi
         \edef\ZH@fontname{\zhfont \ZH@hexbyte\ZH@firstbyte}%
        \expandafter\ifx\csname ZH@\ZH@fontname\endcsname\relax
           \global\expandafter\font\csname ZH@\ZH@fontname\endcsname=\ZH@fontname
        \fi
        \csname ZH@\ZH@fontname\endcsname
        \char\ZH@secondbyte
        \skipzh
      \endgroup}}%
  \catcode"E#1\active}

\ZH@defutftriple{0}\ZH@defutftriple{1}\ZH@defutftriple{2}\ZH@defutftriple{3}
\ZH@defutftriple{4}\ZH@defutftriple{5}\ZH@defutftriple{6}\ZH@defutftriple{7}
\ZH@defutftriple{8}\ZH@defutftriple{9}\ZH@defutftriple{A}\ZH@defutftriple{B}
\ZH@defutftriple{C}\ZH@defutftriple{D}\ZH@defutftriple{E}\ZH@defutftriple{F}

\catcode`@=12

\def\skipzh{\hskip 0em plus 0.2em minus 0.1em}
\def\zhfont{gbsnu}

\endinput
% vim:ft=plaintex
