\documentclass[UTF8,hyperref]{ctexart}
\setmainfont[Ligatures=TeX]{CMU Serif}
\setsansfont[Ligatures=TeX]{CMU Sans Serif}
\setmonofont[Mapping=]{CMU Typewriter Text}
\setCJKmainfont[BoldFont=FandolSong Bold,ItalicFont=KaiTi]{SimSun}

\usepackage[a4paper,centering,margin=1.2in]{geometry}

\CTEXsetup[format=\bfseries\Large]{section}

\renewcommand\refname{参考文档}
\bibliographystyle{plain}

\usepackage{tocbibind}

\usepackage{ragged2e}

\usepackage{booktabs}
\usepackage{longtable}
\usepackage{tabu}

\usepackage{caption}
\captionsetup[table]{font=small,position=top,skip=1ex}

\usepackage{makeidx}
\makeindex

\usepackage{fancyvrb}
\fvset{formatcom=\xeCJKVerbAddon}
\usepackage{shortvrb}
\AtBeginDocument{
  \MakeShortVerb\|
  \MakeShortVerb\"
}

\usepackage{hyperref}
\hypersetup{bookmarksnumbered,pdfstartview=FitH}
\def\tableautorefname{表}

\usepackage{xspace}

\newcommand\pkg[1]{\textsf{#1}\index{#1@\textsf{#1}}}
\newcommand\zhm{\pkg{zhmakeindex}\xspace}
\newcommand\meta[1]{$\langle\textnormal{\itshape#1}\rangle$}
\newcommand\optional[1]{\textnormal{[}#1\textnormal{]}}
\def\defchar#1{\begingroup\lccode`~=`#1\lowercase{\endgroup\def~}}
\newenvironment{syntax}{%
  \quote\ttfamily
  \catcode`<\active  \defchar<##1>{\meta{##1}}
  \catcode`[\active  \defchar[##1]{\optional{##1}}
}{\endquote}
\newcommand\email{\nolinkurl}

\title{\zhm\thanks{版本 1.0} 中文索引处理程序}
\author{刘海洋}
\date{2014 年 2 月 6 日}

\begin{document}

\maketitle

\section{命令行}

\begin{syntax}
\halign{#&#\hfil\cr
zhmakeindex &[-c] [-i] [-o~<ind>] [-q] [-r] [-s~<sty>] [-t~<log>]\cr
            &[-enc~<enc>] [-senc~<senc>] [-strict] [-z~<sort>]\cr
            &[<idx0> <idx1> <idx2> ...]\cr
}
\end{syntax}

\section{简介}

\zhm 是一个通用的中文多级索引处理程序,它从一个或多个输入文件读入索引项,将其
内容按指定的方式分组、排序,然后按格式将整理好的索引输出到文件。索引项可以有 3
个级别(0, 1, 2)的嵌套。\zhm 主要用于 \LaTeX{} 索引的处理,其功能和用法与
\pkg{makeindex}\cite{Rodgers1991} 相似,并支持中文的分组与排序。

输入与输出文件的格式由一个格式文件确定。默认的输入/输出是 ".idx"/".ind" 格式
的,即 \LaTeX{} 格式的索引文件。

如果没有显式指定,第一个输入文件(\meta{idx0})的主文件名将用于确定输出和日志
文件的主文件名。对每个输入文件名 \meta{idx0}, \meta{idx1}, \ldots, \zhm 会首先
查找这个名字的文件;如果找不到且文件名没有后缀,则加上 ".idx" 后缀查找此文件;
如果还找不到文件,\zhm 则会中止。

如果只有一个输入文件,并且没有显式用 "-s" 选项指定格式文件,\zhm 会使用后缀为
".mst" 的默认格式文件(如果存在的话)。

关于如何使用 \pkg{makeindex} 在 \LaTeX{} 文档中处理索引,可以参考陈丕宏较早的
文档 \cite{Chen:1988:IPP},或 Lamport 的文档 \cite{Lamport1987}。\zhm 的在
\LaTeX 中的使用方法与 \pkg{makeindex} 基本一致。

\section{选项说明}

\subsection{与 \pkg{makeindex} 相同的选项}
\label{subsec:oldoption}

\begin{description}
  \item[-c] 压缩索引项排序项前后的空格。默认情况下,排序项中的空格会被保留。
  \item[-i] 从标准输入流("stdin")读入索引项。如果使用了该选项,并且没有使用
    "-o" 选项,则排序后的索引将输出到标准输出流("stdout")。
  \item[-o~\meta{ind}] 设置输出索引文件为 \meta{ind}。如果没有指定该选项,默认
    的输出文件名是第一个输入文件 \meta{idx0} 的主文件名加上 ".ind" 的扩展名。
  \item[-q] 静默模式,不向标准错误流("stderr")显示信息。默认情况下处理过程与
    错误信息会同时在 "stderr" 与日志文件中输出。
  \item[-r] 禁止隐式页码区间构造,要求页码区间必须使用显式区间符号生成。见
    第~\ref{sec:idxsyntax} 节的说明。默认情况下,三个或三个以上连续的页码会自
    动合并为一个页码区间(如 1--5)。
  \item[-s~\meta{sty}] 设置 \meta{sty} 为格式文件。没有默认值。\zhm 会首先在当
    前目录查找格式文件,如果找不到则调用 \TeX{} 发行版的 \pkg{kpathsea} 库在
    TEXMF 树中查找。
  \item[-t~\meta{log}] 设置 \meta{log} 为日志文件。默认情况下,会使用第一个输
    入文件 \meta{idx0} 的主文件名加上 ".ilg" 后缀作为日志文件。
\end{description}

\subsection{\zhm 独有的选项}
\label{subsec:newoption}

\begin{description}
  \item[-enc~\meta{enc}] 设置输入输出文件的编码为 \meta{enc}。可选的编码包括
    "UTF-8", "UTF-16", "GB18030", "GBK" 和 "Big5",不区分大小写。默认使用
    UTF-8 编码。
  \item[-senc~\meta{senc}] 设置读入格式文件的编码为 \meta{senc}。可选的编码与
    "-enc" 选项相同。默认使用 UTF-8 编码。
  \item[-strict] 严格区分不同嵌入命令的页码。默认情况下,在页码区间处理时,会
    将如果页码左区间的嵌入命令与右区间不匹配,会以左区间为准(部分 \LaTeX{} 文
    档会生成右区间命令缺失的索引项);而如果使用 "-strict" 选项,则要求左右区
    间的命令类型必须严格匹配。
  \item[-z~\meta{sort}] 设置中文分组与排序方式为 \meta{sort}。可选的中文分组排
    序方式包括 "pinyin"/"reading", "bihua"/"stroke", "bushou"/"radical"。默认
    值为 "pinyin",即中文按拼音分组排序。有关分组与排序的详细说明见
    第~\ref{sec:sort} 节。
\end{description}

\subsection{未实现的选项}
\label{subsec:unsupportedoption}

\zhm 没有实现 \pkg{makeindex} 的 "-g", "-l", "-p", "-L", "-T" 选项。其中,选项
"-p" 依赖对 \TeX{} 编译的日志文件的解析,可能在未来版本实现;另外几个语言相关
的排序选项("-g" 德文,"-T" 泰文,"-L" 做系统 locale 选择,"-l" 有关西文单词排
序)对中文索引意义较小,不在 \zhm 中实现。

\section{格式文件}

\zhm 的格式文件与标准 \pkg{makeindex} 语法完全相同,内容也基本上相同,同时增加
了少量控制中文分组输出的格式。可以在 \zhm 中使用标准 \pkg{makeindex} 的格式文
件。

格式文件指明了 ".idx" 输入文件和最终输入文件的格式。该文件在当前工作目录或在
TEXMF 树中由 \pkg{kpathsea} 库查找。一个格式文件由一组 \meta{关键字} \meta{属
性} 的列表组成。\meta{关键字} 分为输出和输出两类。\meta{关键字} \meta{属性} 对
儿不要求以任何顺序出现。以字符 \verb"%" 开头的行是注释。\meta{属性} 可以有不同
的类型,字符串是以任意双引号 |"| 或反引号 |`| 界定的串,字符是以单引号 |'| 界
定的单个字符,数字是一个整数。其中,双引号和单引号界定的串和字符,可以使用反斜
线 "\" 符号作为转义符,以得到特殊字符或引号本身;而反引号界定的串是 \zhm 特有
的功能,它不使用转义符。格式文件中没有指定的项目会使用其默认值。

\subsection{与 \pkg{makeindex} 兼容的格式}

\autoref{tab:oldinputstyle} 与\autoref{tab:oldoutputstyle} 的格式是在
\pkg{makeindex} 中有定义,同时也在 \zhm 中有支持的输入、输出格式。\zhm 的这部
分格式选项与 \pkg{makeindex} 意义相同,完全兼容。

\begin{longtabu*} to \linewidth{lll>{\RaggedRight}X}
\caption{与 \pkg{makeindex} 兼容的输入格式}\label{tab:oldinputstyle} \\
  \toprule
  关键字 & 类型 & 默认值 & 意义 \\
  \midrule
\endfirsthead
  \toprule
  关键字 & 类型 & 默认值 & 意义 \\
  \midrule
\endhead
  \bottomrule
\endfoot
  |keyword|  & 字符串 & |"\\indexentry"| & 索引命令 \\
  |arg_open|  & 字符 & |'{'| & 参数的开定界符 \\
  |arg_close|  & 字符 & |'}'| & 参数的闭定界符 \\
  |range_open|  & 字符 & |'('| & 页码范围的开定界符 \\
  |range_close|  & 字符 & |')'| & 页码范围的闭定界符 \\
  |level|  & 字符 & |'!'| & 索引项层次分隔符 \\
  |actual|  & 字符 & |'@'| & (区别于排序项的)实际输出项标志符 \\
  |encap|  & 字符 & "'|'" & 页码和特殊命令指示符 \\
  |quote|  & 字符 & |'"'| & 引号(转义符) \\
  |escape|  & 字符 & |'\\'| & 转义 |quote| 的符号,在它后面的引号不作转义符
解释 \\
\end{longtabu*}

\begin{longtabu*} to \linewidth{lll>{\RaggedRight}X}
\caption{与 \pkg{makeindex} 兼容的输出格式}\label{tab:oldoutputstyle} \\
  \toprule
  关键字 & 类型 & 默认值 & 意义 \\
  \midrule
\endfirsthead
\toprule
关键字 & 类型 & 默认值 & 意义 \\
\midrule
\endhead
\bottomrule
\endfoot
|preamble| &  字符串 & |"\\begin{theindex}\n"| & 索引导言代码\\
|postamble| &  字符串 & |"\n\n\\end{theindex}\n"| & 索引末尾代码\\
|setpage_prefix| &  字符串 & |"\n  \\setcounter{page}{"| & 页码设置前缀\\
|setpage_suffix| &  字符串 & |"}\n"| & 页码设置后缀\\
|group_skip| &  字符串 & |"\n\n  \\indexspace\n"| & 组间垂直间距\\
|headings_flag| &  数字 & |0| & 是否显示分组名标题的旗标\\
|heading_prefix| &  字符串 & |""| & 分组名标题的前缀\\
|heading_suffix| &  字符串 & |""| & 分组名标题的后缀\\
|symhead_positive| &  字符串 & |"Symbols"| & 当旗标 |headings_flag| 为正数时,符号的标题\\
|symhead_negative| &  字符串 & |"symbols"| & 当旗标 |headings_flag| 为负数时,符号的标题\\
|numhead_positive| &  字符串 & |"Numbers"| & 当旗标 |headings_flag| 为正数时,数字的标题\\
|numhead_negative| &  字符串 & |"numbers"| & 当旗标 |headings_flag| 为负数时,数字的标题\\
|item_0| &  字符串 & |"\n  \\item "| & 第 0 级条目间分隔\\
|item_1| &  字符串 & |"\n    \\subitem "| & 第 1 级条目间分隔\\
|item_2| &  字符串 & |"\n      \\subsubitem "| & 第 2 级条目间分隔\\
|item_01| &  字符串 & |"\n    \\subitem "| & 第 0/1 级条目间分隔\\
|item_x1| &  字符串 & |"\n    \\subitem "| & 第 0/1 级条目间分隔,其中第 0 级无页码\\
|item_12| &  字符串 & |"\n      \\subsubitem "| & 第 1/2 级条目间分隔\\
|item_x2| &  字符串 & |"\n      \\subsubitem "| & 第 1/2 级条目间分隔,其中第 1 级无页码\\
|delim_0| &  字符串 & |", "| & 第 0 级条目与页码分隔\\
|delim_1| &  字符串 & |", "| & 第 1 级条目与页码分隔\\
|delim_2| &  字符串 & |", "| & 第 2 级条目与页码分隔\\
|delim_n| &  字符串 & |", "| & 多个页码的分隔\\
|delim_r| &  字符串 & |"--"| & 页码范围符号\\
|delim_t| &  字符串 & |""| & 将插入到在页码列表末尾。如果页码列表为空,此项无
效果。 \\
|encap_prefix| &  字符串 & |"\\"| & 页码特殊指令前缀\\
|encap_infix| &  字符串 & |"{"| & 页码特殊指令中缀\\
|encap_suffix| &  字符串 & |"}"| & 页码特殊指令后缀\\
|page_precedence| &  字符串 & |"rnaRA"| & 不同类型页码的次序,默认值表示小写
  罗马、阿拉伯数字、小写字母、大写罗马、大写字母\\
|suffix_2p| & 字符串 & |""| & 在 2 页的页码范围中代替 |delim_r| 和第二个页码\\
|suffix_3p| & 字符串 & |""| & 在 3 页的页码范围中代替 |delim_r| 和后面的页码\\
|suffix_mp| & 字符串 & |""| & 在更多页的页码范围中代替 |delim_r| 和后面的页码\\
\end{longtabu*}

\subsection{\zhm 特有的格式}

\zhm 定义了一些特有的格式(\autoref{tab:newoutputstyle}),用来控制按笔画数或
部首分组时,分组名的输出格式。

\begin{table}[htbp]
\caption{\zhm 特有的输出格式}\label{tab:newoutputstyle}
\begin{tabu*} to \linewidth{lll>{\RaggedRight}X}
\toprule
关键字 & 类型 & 默认值 & 意义 \\
\midrule
  "stroke_prefix"             & 字符串 & |""| & 笔画数前缀 \\
  "stroke_suffix"             & 字符串 & |" 画"| & 笔画数后缀 \\
  "radical_prefix"            & 字符串 & |""| & 部首前缀 \\
  "radical_suffix"            & 字符串 & |"部"| & 部首后缀 \\
  "radical_simplified_flag"   & 数字 & 1 & 是否输出简化部首的标志 \\
  "radical_simplified_prefix" & 字符串 & |"("| & 简化部首前缀 \\
  "radical_simplified_suffix" & 字符串 & |")"| & 简化部首后缀 \\
\bottomrule
\end{tabu*}
\end{table}

\subsection{未实现的 \pkg{makeindex} 格式}

\autoref{tab:unsupportedinputstyle} 和\autoref{tab:unsupportedoutputstyle} 中
给出的是在标准的 \pkg{makeindex} 中定义,但在 \zhm 中没有实现对应功能的格式。
格式文件中如果出现这些格式,\zhm 将会忽略它们。

\begin{table}[htbp]
\caption{未实现的输入格式}\label{tab:unsupportedinputstyle}
\begin{tabu*} to \linewidth{lll>{\RaggedRight}X}
\toprule
关键字 & 类型 & 默认值 & 意义 \\
\midrule
  |page_compositor| &  字符串 & |"-"| & 复合页码分隔符 \\
\bottomrule
\end{tabu*}
\end{table}

\begin{table}[htbp]
\caption{未实现的输出格式}\label{tab:unsupportedoutputstyle}
\begin{tabu*} to \linewidth{lll>{\RaggedRight}X}
\toprule
关键字 & 类型 & 默认值 & 意义 \\
\midrule
|line_max| &  数字 & |72| & 最大行长度,超出长度会自动折行\\
|indent_space| &  字符串 & |"\t\t"| & 自动折行的缩进\\
|indent_length| &  数字 & |16| & |indent_space| 的长度\\
\bottomrule
\end{tabu*}
\end{table}

\subsection{格式文件示例}

合法的 \pkg{makeindex} 格式文件都是合法的 \zhm 格式文件。\LaTeXe{} 的
\pkg{doc} 包附带的 \path{gind.ist} 和 \path{gglo.ist} 就是两个实际的例子。此
外,\pkg{makeindex} 手册页 \cite{Rodgers1991} 也给出了几个有用的例子。

下面是本文档使用的格式文件,注意其中反引号格式的串是 \zhm 特有的:
\VerbatimInput[numbers=left]{zhmakeindex.mst}

\section{排序细节}
\label{sec:sort}

\subsection{索引项分组}

\zhm 与 \pkg{makeindex} 类似,主要是按第一级索引项的排序项的首个字符分组的。对
最普通的场景,西文是按单词的首字母分组,中文则根据 "-z" 选项
(\ref{subsec:newoption}~节)的不同,选择对应的分组,数字和其他符号单独分组。

\zhm 的前 28 个分组是固定的,分别是数字、符号,以及 A--Z 的 26 个拉丁字母。按
笔画排序时,后面是按总笔画数分组的汉字,共 64 组;按部首笔画排序时,后面是按康
熙字典部首分组的汉字,共 214 组。详细情况见\autoref{tab:group}。

\begin{table}[htbp]
\caption{\zhm 支持的分组方式}\label{tab:group}
\begin{tabu} to \linewidth{lll@{\qquad}>{\RaggedRight}X}
\toprule
"-z" 选项 & 别名 & 前 28 个分组 & 后面的分组 \\
\midrule
"pinyin" & "reading" & 数字、符号、A, \ldots, Z & (无)\\
"bihua" & "stroke"   & 数字、符号、A, \ldots, Z & 1 画、2 画、……、64 画(共 64 组) \\
"bushou" & "radical" & 数字、符号、A, \ldots, Z & 一部、丨部、……、龠部(共 214 组) \\
\bottomrule
\end{tabu}
\end{table}

\zhm 主要是面向中文排版的索引处理,因此对于西文条目,只对没有重音等符号的拉丁
字母能进行正确的分组,对带重音符号的拉丁字母(如 é, ç)、非拉丁字母(如 Σ, Δ,
Ж, Я 等)等则会一律按符号分组。

如果索引项的第一级排序项全部由数字组成,则该项会被分入数字分组;但如果条目的首
个字符是数字,后面还有其他字符,则条目会被计入符号分组。除了阿拉伯数字,\zhm
还将其他 Unicode 数字符号(如罗马数字“Ⅳ”、带圈数字“⑧”)也当作数字处理,但汉字
数码“〇”被看作汉字处理。

当索引项第一级排序项的首个字符不是拉丁字母或汉字,排序项本身也不是纯数字时,则
此索引项就会计入符号分组。

当输出格式变量 "headings_flag" 非零时(参见\autoref{tab:oldoutputstyle}),将
输出分组名称。"headings_flag" 是正数时,以大写字母显示分组名
称;"headings_flag" 是负数时,以小写字母显示分组名称。

当使用汉字特有的分组时,分组名称可以使用格式文件定制(参见%
\autoref{tab:newoutputstyle})。按笔画分组时,分组名为
\begin{syntax}
  <stroke\_prefix><笔画数><stroke\_suffix>
\end{syntax}
按康熙字典部首分组时,如果变量 "radical_simplified_flag" 非零(默认情况),则
分组名由原康熙字典部首与简化字部首组合而成:
\begin{syntax}
  <radical\_prefix><部首><radical\_suffix>\\
  \hspace*{2em}<radical\_simplified\_prefix><简化字部首><radical\_simplified\_suffix>
\end{syntax}
如果变量 "radical_simplified_flag" 是零,则分组名没有简化字部首:
\begin{syntax}
  <radical\_prefix><部首><radical\_suffix>
\end{syntax}

\subsection{索引项排序}

大体上,\zhm 逐字符按字典序对索引项的排序项进行排序,汉字与其他 Unicode 字符一
样,按单个字符比较。排序时使用的字符串比较有一些特殊规则:
\begin{itemize}
  \item 如果字符串只包含阿拉伯数字,则首先按数字大小比较,数字相等时再按字典序
    比较。
  \item 以符号开头的字符串总是先于排在以数字开头的串(即使符号的 Unicode 码在
    数字之后),而这又先于纯数字的排序项和以字母开头的串。
  \item 在比较两个字符串时,\zhm 首先忽略字母大小写进行比较,如果此时结果相
    等,再按区分大小写进行比较,将大写字母排在小写字母之前。
\end{itemize}

\pkg{makeindex} 有一个 "-l" 选项忽略条目中的空格,按所有非空白符比较所有条目
(\ref{subsec:unsupportedoption}~节)。\zhm 未提供此功能。

按 "-z" 选项(\ref{subsec:newoption}~节)的不同,汉字可以使用不同的排序方式,
如\autoref{tab:sort} 所示。
\begin{table}[htbp]
\caption{\zhm 支持的中文排序方式}\label{tab:sort}
\begin{tabu} to \linewidth{ll>{\RaggedRight}X}
\toprule
"-z" 选项 & 别名 & 意义 \\
\midrule
"pinyin" & "reading" & 汉字按常用读音的拼音排序。 \\
"bihua" & "stroke" & 汉字按笔画数和笔顺排序。 \\
"bushou" & "radical" & 汉字按康熙字典部首和除部首笔画数排序。 \\
\bottomrule
\end{tabu}
\end{table}

使用读音排序时,没有读音数据的字符(包括生僻字)一律排在有读音的字符之前,汉字
按其最常用读音比较,读音相同汉字的按 Unicode 编码排序。使用笔画数和笔顺排序
时,笔画数小的排在前面,笔画数相同的,按横、竖、撇、点(捺)、折的顺序逐笔画比
较,仍然相同的按 Unicode 编码排序;生僻汉字没有笔顺信息的,排在同笔画数有笔顺
的字后面。使用部首和除部首笔画数排序时,部首按康熙字典 214 部首顺序排列,部首
和笔画数相同的按 Unicode 编码排序。

\subsection{页码排序与合并}

内容完全相同而页码不同的索引项,会在排序时合并为一项。同一索引项的所有页码会按
前后次序排序、去重,并按要求合并为页码区间。

\zhm 能识别不同类型的页码格式,包括阿拉伯数字(1, 2, 3, \ldots)、大小写罗马数
字(I, II, III, \ldots)、大小写拉丁字母(a, b, c, \ldots)。不同类型的页码按
照 "page_precedence" 变量的设置(\autoref{tab:oldoutputstyle})区分前后次序。
默认情况下,不同类型的页码次序是 "rnaRA",即小写罗马数字的页码排在最前,然后依
次是阿拉伯数字、小写拉丁字母、大写罗马数字、大写拉丁字母的页码。

目前,页码的类型判别比较粗糙,仅以页码的第一个字符判断类型。当页码以字母 "i",
"v", "x", "l", "c", "d", "m" 起始时,即被识别为罗马数字,其他字母被识别为拉丁
字母页码。

\zhm 支持页码区间,在索引项中显式指定区间,如对输入
\begin{verbatim}
\indexentry{foo|(}{5}
\indexentry{foo|)}{10}
\end{verbatim}
则 "foo" 项会输出页码 5--10。连续的区间、区间内的零散页码会被合并为一个大的区
间,如对输入
\begin{verbatim}
\indexentry{foo|(}{5}
\indexentry{foo}{6}
\indexentry{foo|)}{10}
\indexentry{foo|(}{11}
\indexentry{foo|)}{13}
\end{verbatim}
则 "foo" 项会输出页码 5--13。

如果没有使用 "-r" 选项(\ref{subsec:oldoption}~节),零散的页码也会与区间一起
合并,3 页或以上连续的页码也会被合并为区间。例如输入
\begin{verbatim}
\indexentry{foo}{i}
\indexentry{foo}{ii}
\indexentry{foo}{iii}
\indexentry{foo|(}{5}
\indexentry{foo|)}{7}
\indexentry{foo}{8}
\end{verbatim}
则 "foo" 项会输出页码 i--iii, 5--8。

在排序合并页码时,\zhm 会区分不同数字类型的页码;如果页码还有特殊命令修饰,则
还会区分不同的修饰命令。


\section{索引项输入语法}
\label{sec:idxsyntax}

\section{与 \pkg{makeindex} 的比较}

\section{版权与许可}
\index{版权}\index{许可}

版权所有:2014 年,刘海洋 \email{leoliu.pku@gmail.com}

\index{LPPL}
本作品可在《the \LaTeX{} Project Public License》1.3 或更高版本的条件下发布与
修改。最新版本的 LPPL 许可证可以在
\begin{quote}
  \url{http://www.latex-project.org/lppl.txt}
\end{quote}
下载;该许可证同时也包含在所有最新的 \LaTeX{} 发行版中。

本作品目前处于 LPPL 维护状态“author-maintained”。

当前维护者是刘海洋。

\index{源文件}
本作品包括 \zhm 的程序及文档,由如下源文件:
\begin{verbatim}
build-dist.cmd
input.go
main.go
numberedreader.go
output.go
radicalstrokes.go
radical_collator.go
readings.go
reading_collator.go
sorter.go
strokes.go
stroke_collator.go
style.go
style_test.go
doc/zhmakeindex.bib
doc/zhmakeindex.mst
doc/zhmakeindex.tex
kpathsea/dynamic_other.go
kpathsea/dynamic_windows_386.go
kpathsea/kpathsea.go
maketables/make-table.cmd
maketables/maketables.go
\end{verbatim}
以及编译源文件得到的二进制文件 \path{zhmakeindex.exe} 或 \path{zhmakeindex}、
PDF 文档 \path{zhmakeindex.pdf} 组成。

\index{Unicode}
大部分汉字数据来自 Unicode 项目(\url{http://www.unicode.org/}):
\begin{verbatim}
maketables/CJKRadicals.txt
maketables/Unihan_DictionaryLikeData.txt
maketables/Unihan_RadicalStrokeCounts.txt
maketables/Unihan_Readings.txt
\end{verbatim}

\index{海峰五笔}
部分字形数据来自海峰五笔项目(\url{http://okuc.net/sunwb/}):
\begin{verbatim}
maketables/sunwb_strokeorder.txt
\end{verbatim}

\bibliography{zhmakeindex}

\printindex

\end{document}

vim:tw=78
