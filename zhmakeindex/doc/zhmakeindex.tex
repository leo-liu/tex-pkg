\documentclass[UTF8,hyperref]{ctexart}

\usepackage[a4paper,centering,margin=1.5in]{geometry}

\CTEXsetup[format=\bfseries\Large]{section}

\usepackage{makeidx}
%\makeindex

\newcommand\pkg{\textsf}
\newcommand\meta[1]{$\langle\textnormal{\itshape#1}\rangle$}
\newcommand\email{\nolinkurl}

\title{\pkg{zhmakeindex}\thanks{版本 1.0} 中文索引处理程序}
\author{刘海洋}
\date{2014 年 2 月 6 日}

\begin{document}

\maketitle

\section{命令行}

\begin{quote}
\ttfamily
\catcode`<\active  \def<#1>{\meta{#1}}
\halign{#&#\hfil\cr
zhmakeindex &-enc~<enc> -i -o~<output> -q -r -s~<style>\cr
            &-senc~<senc> -strict -t~<log> -z~<sort>\cr
            &<idx1> <idx2> ...\cr}
\end{quote}

注:除输入文件外,所有选项均可省略。

\section{简介}

\pkg{zhmakeindex} 是一个通用的中文索引处理程序,它从一个或多个输入文件读入索引
项,将其内容按指定的方式分组、排序,然后输出整理好的索引文件。
\pkg{zhmakeindex} 主要用于 \LaTeX{} 的索引生成,其功能和用法与 \pkg{makeindex}
相似,并支持中文的分组与排序。

\section{选项说明}

\begin{description}
  \item[-enc~\meta{enc}] 设置输入输出文件的编码为 \meta{enc}。
  \item[-senc~\meta{senc}] 设置读入格式文件的编码为 \meta{senc}。
  \item[-strict] 严格区分不同 encapsulated 命令的页码。
  \item[-z~\meta{sort}] 设置中文排序分组方式为 \meta{sort}。
\end{description}

\section{格式文件}

\section{顺序}

\section{特殊效果}

\section{与 \pkg{makeindex} 的比较}

\section{版权与许可}

版权所有:2014 年,刘海洋 \email{leoliu.pku@gmail.com}

本作品可在《the \LaTeX{} Project Public License》1.3 或更高版本的条件下发布与
修改。最新版本的 LPPL 许可证可以在
\begin{quote}
  \url{http://www.latex-project.org/lppl.txt}
\end{quote}
下载;该许可证同时也包含在所有最新的 \LaTeX{} 发行版中。

本作品目前处于 LPPL 维护状态“author-maintained”。

当前维护者是刘海洋。

本作品包括 \pkg{zhmakeindex} 的程序及文档,由如下源文件:
\begin{verbatim}
dist.cmd
examples
input.go
main.go
numberedreader.go
output.go
radicalstrokes.go
radical_collator.go
readings.go
reading_collator.go
sorter.go
strokes.go
stroke_collator.go
style.go
style_test.go
doc/zhmakeindex.tex
kpathsea/dynamic_other.go
kpathsea/dynamic_windows_386.go
kpathsea/kpathsea.go
maketables/make-table.cmd
maketables/maketables.go
\end{verbatim}
以及编译源文件得到的二进制文件 \path{zhmakeindex.exe}、PDF 文档
\path{zhmakeindex.pdf} 组成。

大部分汉字数据来自 Unicode 项目(\url{http://www.unicode.org/}):
\begin{verbatim}
maketables/CJKRadicals.txt
maketables/Unihan_DictionaryLikeData.txt
maketables/Unihan_RadicalStrokeCounts.txt
maketables/Unihan_Readings.txt
\end{verbatim}

部分字形数据来自海峰五笔项目(\url{http://okuc.net/sunwb/}):
\begin{verbatim}
maketables/sunwb_strokeorder.txt
\end{verbatim}

\end{document}

vim:tw=78
