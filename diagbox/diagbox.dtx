% \iffalse meta-comment
%
% Copyright (C) 2011 by Leo Liu <leoliu.pku@gmail.com>
% ---------------------------------------------------------------------------
% This work may be distributed and/or modified under the
% conditions of the LaTeX Project Public License, either version 1.3
% of this license or (at your option) any later version.
% The latest version of this license is in
%   http://www.latex-project.org/lppl.txt
% and version 1.3 or later is part of all distributions of LaTeX
% version 2005/12/01 or later.
%
% This work has the LPPL maintenance status `maintained'.
%
% The Current Maintainer of this work is Leo Liu.
%
% This work consists of the files diagbox.dtx and diagbox.ins
% and the derived filebase diagbox.sty.
%
% \fi
%
% \iffalse
%<*driver>
\ProvidesFile{diagbox.dtx}
%</driver>
%<package>\NeedsTeXFormat{LaTeX2e}[1999/12/01]
%<package>\ProvidesPackage{diagbox}
%<*package>
    [2011/11/21 v1.0 Making table heads with diagonal lines]
%</package>
%
%<*driver>
\documentclass[a4paper]{ltxdoc}
\AtBeginDocument{
  \DeleteShortVerb{\|}
  \MakeShortVerb{\"}}
\usepackage{diagbox}
\usepackage{geometry}
\usepackage{fvrb-ex}
\usepackage[UTF8,hyperref]{ctex}
\usepackage[pdfstartview=FitH]{hyperref}
\EnableCrossrefs
\CodelineIndex
\RecordChanges
\begin{document}
  \DocInput{diagbox.dtx}
  \PrintChanges
  \PrintIndex
\end{document}
%</driver>
% \fi
%
% \CheckSum{201}
%
% \CharacterTable
%  {Upper-case    \A\B\C\D\E\F\G\H\I\J\K\L\M\N\O\P\Q\R\S\T\U\V\W\X\Y\Z
%   Lower-case    \a\b\c\d\e\f\g\h\i\j\k\l\m\n\o\p\q\r\s\t\u\v\w\x\y\z
%   Digits        \0\1\2\3\4\5\6\7\8\9
%   Exclamation   \!     Double quote  \"     Hash (number) \#
%   Dollar        \$     Percent       \%     Ampersand     \&
%   Acute accent  \'     Left paren    \(     Right paren   \)
%   Asterisk      \*     Plus          \+     Comma         \,
%   Minus         \-     Point         \.     Solidus       \/
%   Colon         \:     Semicolon     \;     Less than     \<
%   Equals        \=     Greater than  \>     Question mark \?
%   Commercial at \@     Left bracket  \[     Backslash     \\
%   Right bracket \]     Circumflex    \^     Underscore    \_
%   Grave accent  \`     Left brace    \{     Vertical bar  \|
%   Right brace   \}     Tilde         \~}
%
%
% \changes{v1.0}{2011/11/21}{初始版本}
%
% \DoNotIndex{\newcommand, \newenvironment, \@nameuse, \@pkgextension,
% \@reserveda, \@tfor, \begin, \begingroup, \box, \csname, \def, \define@key,
% \dimexpr, \do, \dp, \edef, \else, \end, \endcsname, \endgroup, \endinput,
% \expandafter, \fi, \hbox, \ht, \ifcsname, \ifdim, \ifx, \Line, \makebox,
% \newbox, \newdimen, \p@, \PackageError, \pkg, \put, \relax, \RequirePackage,
% \setbox, \setkeys, \setlength, \strip@pt, \unexpanded, \unitlength,
% \vcenter, \wd, \xdef, \z@}
%
% \providecommand*{\pkg}{\textsf}
% \GetFileInfo{diagbox.dtx}
% \title{\hypertarget{English}{\pkg{diagbox} Package (\fileversion)}
%    \makebox[0pt]{\hspace{5cm}\hyperlink{Chinese}{\textsf{中文版}}}\\
%    Making Table Heads with Diagonal Lines}
% \author{Leo Liu \\ \path{leoliu.pku@gmail.com}}
% \date{\filedate}
%
% \maketitle
%
% \section{Introduction}
%
% \pkg{diagbox} is a replacement of old \pkg{slashbox} package. I write this
% package simply because that \pkg{slashbox} is not available in \TeX\ Live
% for licening problems. \pkg{slashbox} has no explicit license information
% available, but \pkg{diagbox} is under LPPL.
%
% \pkg{diagbox} is a modern alternative of \pkg{slashbox}. I changed the
% user interface to use a key-value syntax, get rid of some restrictions of
% \pkg{slashbox}, use \pkg{pict2e} to draw diagonal lines. \pkg{diagbox}
% package also provides compatible macros of \pkg{slashbox}, but the result is
% a little different.
%
% To use \pkg{diagbox}, \eTeX{} is needed. And \pkg{diagbox} requires
% \pkg{pict2e} package.
%
% \section{Usage}
%
%
% \DescribeMacro{\diagbox}
% "\diagbox" is the main command. It takes two arguments, to produce a box
% with a diagonal line from north west to south east.
%
% For example:\\
% \begin{SideBySideExample}[frame=single,numbers=left,xrightmargin=.45\linewidth]
% \begin{tabular}{|l|ccc|}
% \hline
% \diagbox{Time}{Day} & Mon & Tue & Wed \\
% \hline
% Morning   & used & used &      \\
% Afternoon &      & used & used \\
% \hline
% \end{tabular}
% \end{SideBySideExample}
%
% \medskip
% "\diagbox" can take an optional argument in key-value syntax to specify the
% width and height of the box, the direction of the diagonal line, and the
% trimming margin:
% \begin{description}
% \item[width] Specify the width of the box explicitly.
% \item[height] Specify the height of the box explicitly.
% \item[dir] Specify the direction of the diagonal line. The value can be "SE"
% and "NE". Default value is "SE" (from north west to south east).
% \item[trim] Specify the margin to be trimmed. The value can be "l", "r", and
% "lr", "rl".
% \end{description}
%
% Here is a more complex example:
% \begin{Example}[frame=single,numbers=left]
% \begin{tabular}{|@{}l|c|c|c@{}|}
% \hline
% \diagbox[width=5em,trim=l]{Time}{Day} & Mon & Tue & Wed\\
% \hline
% Morning   & used & used & used\\
% \hline
% Afternoon &      & used & \diagbox[dir=NE,height=2em,trim=r]{A}{B} \\
% \hline
% \end{tabular}
% \end{Example}
%
% \bigskip
% \DescribeMacro{\backslashbox}
% "\backslashbox" works as "\diagbox", but it takes two optional arguments to
% specify the "width" and "trim" options.
%
% \DescribeMacro{\slashbox}
% "\slashbox" works as "\diagbox[dir=NE]", and takes two optional arguments to
% specify the "width" and "trim" options.
%
% The syntax of "\slashbox" and "\backslashbox" comes from \pkg{slashbox}
% packages. \pkg{diagbox} package emulates \pkg{slashbox} and also prevents
% \pkg{slashbox} to be loaded. However, the results of the two packages are
% a little different. These two commands are for compatibility only, it is
% better to use "\diagbox" instead.
%
%
% \section{Bugs and TODO}
%
% I will provide a new command to produce a three-part box with two diagonal
% lines in later version.
%
%
%
%
% \title{\hypertarget{Chinese}{\pkg{diagbox} 宏包(\fileversion)}
%    \makebox[0pt]{\hspace{5cm}\hyperlink{English}{\textsf{English Version}}}\\
%    制做斜线表头}
% \author{刘海洋 \\ \path{leoliu.pku@gmail.com}}
% \date{\filedate}
%
% \maketitle
%
%
% \section{简介}
%
% \pkg{diagbox} 设计用来代替旧的 \pkg{slashbox} 宏包。编写这个宏包的缘起是
% \pkg{slashbox} 因为缺少明确的自由许可信息,被 \TeX\ Live 排除。这个宏包是在
% LPPL 协议下发行的。
%
% \pkg{diagbox} 是 \pkg{slashbox} 宏包的一个较现代的版本。这里我采用了新的
% key-value 式语法参数,去除了 \pkg{slashbox} 原有的一些长度限制,并调用
% \pkg{pict2e} 宏包画斜线。\pkg{diagbox} 除了提供自己的新命令,也提供了
% \pkg{slashbox} 原有的两个命令,语法不变,编译结果略有区别。
%
% \pkg{diagbox} 依赖 \eTeX{} 扩展(这在目前总是可用的),依赖 \pkg{pict2e} 宏
% 包。
%
% \section{用法说明}
%
% \DescribeMacro{\diagbox}
% "\diagbox" 是宏包提供的主要命令。它带有两个必选参数,表示要生成斜线表头的两
% 部分内容。默认斜线是从西北到东南方向的。
%
% 例如:\\
% \begin{SideBySideExample}[frame=single,numbers=left,xrightmargin=.45\linewidth]
% \begin{tabular}{|l|ccc|}
% \hline
% \diagbox{Time}{Day} & Mon & Tue & Wed \\
% \hline
% Morning   & used & used &      \\
% Afternoon &      & used & used \\
% \hline
% \end{tabular}
% \end{SideBySideExample}
%
% \medskip
% "\diagbox" 还可以带一个可选参数,里面用 key-value 的语法设置宽度、方向等更多
% 的选项:
% \begin{description}
% \item[width] 明确指定盒子的总宽度。
% \item[height] 明确指定盒子的总高度。
% \item[dir] 指定斜线方向。可以取 "NE"(划向东北)和 "SE"(划向东南)两个值,
% 默认值是 "SE"。
% \item[trim] 设置左边界或右边界不计算额外的空白,可以取值为 "l", "r", "lr" 或
% "rl"。这个选项在列格式包含 "@{}" 时将会有用。
% \end{description}
%
% 一个更繁杂的例子:
% \begin{Example}[frame=single,numbers=left]
% \begin{tabular}{|@{}l|c|c|c@{}|}
% \hline
% \diagbox[width=5em,trim=l]{Time}{Day} & Mon & Tue & Wed\\
% \hline
% Morning   & used & used & used\\
% \hline
% Afternoon &      & used & \diagbox[dir=NE,height=2em,trim=r]{A}{B} \\
% \hline
% \end{tabular}
% \end{Example}
%
% \bigskip
% \DescribeMacro{\backslashbox}
% "\backslashbox" 基本功能与 "\diagbox" 类似。它带有两个可选参数,分别表示
% "\diagbox" 中的 "width" 与 "trim" 选项。
%
% \DescribeMacro{\slashbox}
% "\slashbox" 基本功能与 "\diagbox[dir=NE]" 类似。它也带有两个可选参数,表示
% "\diagbox" 中的 "width" 和 "tirm" 选项。
%
% "\slashbox" 与 "\backslashbox" 的语法来自 \pkg{slashbox} 宏包。\pkg{diagbox}
% 宏包模拟了 \pkg{slashbox} 宏包的功能,并禁止 \pkg{slashbox} 再被调用。这两个
% 命令仅在旧文档中作为兼容命令使用。实际中使用 "\diagbox" 更为方便。
%
%
% \section{已知问题和未来版本}
%
% 未来版本会增加带有两条斜线的表头命令。这部分代码是以前写过的,正在整理中。
%
% \StopEventually{}
%
%
% \clearpage
% \section{Implementation / 代码实现}
%
% \iffalse
%<*package>
% \fi
%
%
% 使用 key-value 界面。
%    \begin{macrocode}
\RequirePackage{keyval}
%    \end{macrocode}
% 绘图依赖 \pkg{pict2e} 宏包。
%    \begin{macrocode}
\RequirePackage{pict2e}
\newbox\diagbox@boxa
\newbox\diagbox@boxb
\newdimen\diagbox@wd
\newdimen\diagbox@ht
\newdimen\diagbox@sepl
\newdimen\diagbox@sepr
\define@key{diagbox}{width}{%
  \setlength{\diagbox@wd}{#1}}
\define@key{diagbox}{height}{%
  \setlength{\diagbox@ht}{#1}}
\define@key{diagbox}{trim}{%
  \@tfor\@reserveda:=#1\do{%
    \setlength{\csname diagbox@sep\@reserveda\endcsname}{\z@}}}
\define@key{diagbox}{dir}{%
  \ifcsname diagbox@pict@#1\endcsname
    \def\diagbox@pict@content{\@nameuse{diagbox@pict@#1}}%
  \else
    \PackageError{diagbox}{Unknown direction `#1'.}{SE and NE is supported.}%
  \fi}
%    \end{macrocode}
%
% \begin{macro}{\diagbox@pict}
% 这是带斜线的盒子本身。由一个 "picture" 环境实现。
%    \begin{macrocode}
\def\diagbox@pict{%
  \unitlength\p@
  \begin{picture}
    (\strip@pt\dimexpr\diagbox@wd-\diagbox@sepl-\diagbox@sepr\relax,\strip@pt\diagbox@ht)
    (\strip@pt\diagbox@sepl,0)
      \diagbox@pict@content
  \end{picture}}
%    \end{macrocode}
% \end{macro}
%
%
% \begin{macro}{\diagbox@pict@SE}
% 方向为 "SE" 的斜线盒子内容。
%    \begin{macrocode}
\def\diagbox@pict@SE{%
  \put(0,0) {\makebox(0,0)[bl]{\box\diagbox@boxa}}
  \put(\strip@pt\diagbox@wd,\strip@pt\diagbox@ht) {\makebox(0,0)[tr]{\box\diagbox@boxb}}
  \Line(0,\strip@pt\diagbox@ht)(\strip@pt\diagbox@wd,0)}
%    \end{macrocode}
% \end{macro}
%
%
% \begin{macro}{\diagbox@pict@NE}
% 方向为 "NE" 的斜线盒子内容。
%    \begin{macrocode}
\def\diagbox@pict@NE{%
  \put(0,\strip@pt\diagbox@ht) {\makebox(0,0)[tl]{\box\diagbox@boxa}}
  \put(\strip@pt\diagbox@wd,0) {\makebox(0,0)[br]{\box\diagbox@boxb}}
  \Line(0,0)(\strip@pt\diagbox@wd,\strip@pt\diagbox@ht)}
%    \end{macrocode}
% \end{macro}
%
%
% \begin{macro}{\diagbox}
% 主要的用户命令。三个参数,分别为 key-value 格式的可选项、左半边内容、右半边
% 内容。这个命令的主要内容是读入参数并计算斜线盒子的大小。
%    \begin{macrocode}
\newcommand\diagbox[3][]{%
  \begingroup
  \diagbox@wd=\z@
  \diagbox@ht=\z@
  \diagbox@sepl=\tabcolsep
  \diagbox@sepr=\tabcolsep
  \setkeys{diagbox}{dir=SE,#1}%
  \setbox\diagbox@boxa=\hbox{%
    \begin{tabular}{l@{}}#2\end{tabular}}%
  \setbox\diagbox@boxb=\hbox{%
    \begin{tabular}{@{}r}#3\end{tabular}}%
  \ifdim\diagbox@wd=\z@
    \ifdim\wd\diagbox@boxa>\wd\diagbox@boxb
      \diagbox@wd=\dimexpr2\wd\diagbox@boxa+\diagbox@sepl+\diagbox@sepr\relax
    \else
      \diagbox@wd=\dimexpr2\wd\diagbox@boxb+\diagbox@sepl+\diagbox@sepr\relax
    \fi
  \fi
  \ifdim\diagbox@ht=\z@
    \diagbox@ht=\dimexpr\ht\diagbox@boxa+\dp\diagbox@boxa+\ht\diagbox@boxb+\dp\diagbox@boxb\relax
  \fi
  $\vcenter{\hbox{\diagbox@pict}}$%
  \endgroup}
%    \end{macrocode}
% \end{macro}
%
% 以下代码用来模拟 \pkg{slashbox} 宏包的功能。
%
% 禁止读入 \pkg{slashbox}。
%    \begin{macrocode}
\expandafter\xdef\csname ver@slashbox.\@pkgextension\endcsname{9999/99/99}
%    \end{macrocode}
%
%
% \begin{macro}{\slashbox}
% 模拟 "\slashbox"。
%    \begin{macrocode}
\def\slashbox{%
  \def\diagbox@slashbox@options{dir=NE,}%
  \slashbox@}
%    \end{macrocode}
% \end{macro}
%
%
% \begin{macro}{\backslashbox}
% 模拟 "\backslashbox"。
%    \begin{macrocode}
\def\backslashbox{%
  \def\diagbox@slashbox@options{dir=SE,}%
  \slashbox@}
%    \end{macrocode}
% \end{macro}
%
%
% \begin{macro}{\slashbox@}
%    \begin{macrocode}
\newcommand\slashbox@[1][]{%
  \ifx\relax#1\relax\else
    \edef\diagbox@slashbox@options{%
      \unexpanded\expandafter{\diagbox@slashbox@options}%
      \unexpanded{width=#1,}}%
  \fi
  \slashbox@@}
%    \end{macrocode}
% \end{macro}
%
%
% \begin{macro}{\slashbox@@}
%    \begin{macrocode}
\newcommand\slashbox@@[3][]{%
  \edef\diagbox@slashbox@options{%
    \unexpanded\expandafter{\diagbox@slashbox@options}%
    \unexpanded{trim=#1,}}%
  \expandafter\diagbox\expandafter[\diagbox@slashbox@options]{#2}{#3}}
\endinput
%    \end{macrocode}
% \end{macro}
%
%
% \iffalse
%</package>
% \fi
%
% \Finale
\endinput
